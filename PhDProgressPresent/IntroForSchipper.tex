%%%%%%%%%%%%%%%%%%%%%%%%%%%%%%%%%%%%%%%%%%%%%%%%%%%%%%%%%%%%%%%%%%%%%%
% Overleaf (WriteLaTeX) Example: Molecular Chemistry Presentation
%
% Source: http://www.overleaf.com
%
% In these slides we show how Overleaf can be used with standard 
% chemistry packages to easily create professional presentations.
% 
% Feel free to distribute this example, but please keep the referral
% to overleaf.com
% 
%%%%%%%%%%%%%%%%%%%%%%%%%%%%%%%%%%%%%%%%%%%%%%%%%%%%%%%%%%%%%%%%%%%%%%
% How to use Overleaf: 
%
% You edit the source code here on the left, and the preview on the
% right shows you the result within a few seconds.
%
% Bookmark this page and share the URL with your co-authors. They can
% edit at the same time!
%
% You can upload figures, bibliographies, custom classes and
% styles using the files menu.
%
% If you're new to LaTeX, the wikibook is a great place to start:
% http://en.wikibooks.org/wiki/LaTeX
%
%%%%%%%%%%%%%%%%%%%%%%%%%%%%%%%%%%%%%%%%%%%%%%%%%%%%%%%%%%%%%%%%%%%%%%

\documentclass{beamer}

% For more themes, color themes and font themes, see:
% http://deic.uab.es/~iblanes/beamer_gallery/index_by_theme.html
%
\mode<presentation>
{
  \usetheme{Madrid}       % or try default, Darmstadt, Warsaw, ...
  \usecolortheme{default} % or try albatross, beaver, crane, ...
  \usefonttheme{serif}    % or try default, structurebold, ...
  \setbeamertemplate{navigation symbols}{}
  \setbeamertemplate{caption}[numbered]
} 

\usepackage[english]{babel}
\usepackage[utf8x]{inputenc}
\usepackage{chemfig}
\usepackage[version=3]{mhchem}
%allows for strikeout:
\usepackage[normalem]{ulem}
% On Overleaf, these lines give you sharper preview images.
% You might want to `comment them out before you export, though.
\usepackage{pgfpages}
\usepackage{hyperref}
\hypersetup{
    colorlinks=false,
    linkcolor=blue,
    filecolor=magenta,      
    urlcolor=cyan,
}
\usepackage{ amssymb }
%\usepackage[abbr]{harvard}
\usepackage{natbib}

\pgfpagesuselayout{resize to}[%
  physical paper width=8in, physical paper height=6in]

% Here's where the presentation starts, with the info for the title slide
\title[Phd Status]{Status Update for PhD Progress}
\author{Tyler Hoppenfeld}
\institute{UC Davis}
\date{\today}

\newtheorem{thm}{Theorm}
\newtheorem{prop}{Proposition}
\newtheorem{conj}{Conjecture}

\newtheorem{deff}{Definition}
\usepackage{graphicx}
\graphicspath{ {../} }
\begin{document}

\begin{frame}
  \titlepage
\end{frame}

% These three lines create an automatically generated table of contents.
\begin{frame}{Outline for Today}
  \tableofcontents
\end{frame}

\section{Goals}

\subsection{Oral in ~3 Months?}

\begin{frame}{Oral in ~3 Months?}
	\begin{itemize}
		\item Discuss timing for oral exam
	\end{itemize}
\end{frame}

\subsection{Seek MD specific feedback}

\begin{frame}{Seek MD specific feedback}
	\begin{itemize}
		\item I am close to a circulation-ready draft, I intend to seek an MD specific committee member via Stanford connections.
	\end{itemize}
\end{frame}


\subsection{Seek game-theory specific feedback}

\begin{frame}{Seek game-theory specific feedback}
	\begin{itemize}
		\item That is what I am doing right now, talking to you.
	\end{itemize}
\end{frame}





\subsection{3 \emph{Good} Papers}

\begin{frame}{3 \emph{Good} Papers}
	\begin{itemize}
		\item I want to finish up reasonably soon, but definitely don't want to skimp on paper quality
	\end{itemize}
\end{frame}

\begin{frame}{Outline for Today}
	\tableofcontents
\end{frame}

\section{Research Agenda}
\subsection{Animating Conjecture}
\begin{frame}{Animating Conjecture}
	Early contracting will result from a matching process where there is lots of scope for strategy 
\end{frame}

\subsection{Pieces of this}
\subsubsection{Characterize early contracting}
\begin{frame}{Characterize early contracting}
	\begin{itemize}
		\item This is my ``Unraveling'' line of work
		\item In this line of work I have a model where there is a self-fulfilling process of early contracting that is welfare destroying (the state where all participate in the organized market is pareto preferred to equilibrium) 
	\end{itemize}
\end{frame}

\subsubsection{Characterize when matching games have `large' cores}
\begin{frame}{Characterize when matching games have `large' cores}
	This is my  most current work, which I will reprise in detail below
\end{frame}


\subsubsection{Characterize the strategic interactions within the matching game}
\begin{frame}{Characterize the strategic interactions within the matching game}
	I believe this to be an intractable problem in the general case but I hope to solve it in a simpler model of preferences
\end{frame}

\subsubsection{Formally combine these ideas}
\begin{frame}{Formally combine these ideas}
	\begin{itemize}
		\item This is ambitious--I'd like to say something in general about matching markets where the choice to participate is a strategic option.  
		\item I believe this is also the biggest gap in market design. MD mechanisms are embedded in a larger game that the designers do not control, and thats not well enough covered in the literature yet
	\end{itemize}
	 
\end{frame}

\begin{frame}{Outline for Today}
	\tableofcontents
\end{frame}

\section{ Summary of `Large Core' Work}
\begin{frame}{Background}
    Theory tells us that there is no matching mechanism that reliably produces a stable outcome wherein both sides have the incentive to accrurately report their preferences \citep{Roth1985}, however in practice many matching markets persist with both sides (apparently) reporting their preferences honestly  \citep{Roth1991}. This result can be explained by the fact that in many of these markets the set of stable matchings is quite small, and under some realistic assumptions only agents with multiple potential stable matches have an incentive to lie about their preferences.  For example, in the National Resident Matching Program (NRMP) matching of recently graduated physicians to resident positions, all but approximately 10 of 20,000 participants have a unique partner in the set of all stable matchings \citep{Roth1999a}, and so in all but a very small number of cases any misrepresentation of preferences would lead to an identical or less-preferred outcome for the agent misrepresenting their preferences. 
\end{frame}


\begin{frame}{Background}
    Roth and Peranson posit that the small set of stable matchings (SSM) of these markets is driven by the correlated nature of preferences, and by the fact that for highly correlated preferences, the SSM is very small.  In a parallel argument, they note that finite preference lists, as is the case with the NRMP, lead to a very small SSM as well.  They use simulation to demonstrate these results, and (\cite{Immorlica2005} and then \cite{Kojima2009} build on this with the theoretical result that as the number of market participants grows, the SSM becomes arbitrarily small as a fraction of market size. This theoretical work takes short preference lists to be a feature of preferences rather than a feature of the market mechanism, and uses a highly restrictive preference structure to accomplish that. 
\end{frame}


\begin{frame}{Background}
    The primary restriction on preferences that  I relax is on the existence of broadly acceptable `elite' applicants. In prior literature where hospitals form $k$ length preference lists over $n$ doctors and preference lists of length $k$,  for any $\epsilon_a$, no more than $k/\epsilon_a$ doctors can be acceptable to $n\epsilon_a$ hospitals. That is to say that as the market grows, nearly all doctors are unacceptable to nearly all hospitals. ).  

    Here I examine short rank lists in a setting that allows for more realistic preference formation. I construct preferences where all participants on each side in a matching market broadly share preferences over their potential matches, requiring short rank-lists does not introduce strategic behavior, and the fraction of agents with multiple stable matches has a positive limit as the market becomes infinitely large.
\end{frame}


\begin{frame}{Technical Summary of `Large Core' Work}
	the following is a technical summary of my work on the size of the set of stable matches
\end{frame}

\begin{frame}{Model Definitions}
	\begin{itemize}
		\item  $D$ Doctors and $H$ Hospitals: $U = D \cup H$.  
		\item Doctors have preferences $\succ_{D}$ over $H \cup \{d\}$ and hospitals likewise
		\item A matching $\phi$ is a mapping from $U$ onto itself such that 
		\begin{itemize}
			\item for every $d$, $|\phi(d)| = 1$, $\phi(d) \in H \cup \{c\} $ and for every  $h$, $|\phi(h)| = 1$, $\phi(h) \in D \cup \{h\}$
			\item $\phi(h) = d$ iff $\phi(d) = h$
		\end{itemize}
	\item $\phi$ is blocked by a pair $\{d,h\}$ if $h \succ_d \phi(d)$ and $d \succ_h \phi(h)$ 
	\item It is individually rational if $\forall a \in U ,\phi(a) \succeq a$ 
	\item $\phi$ is stable if it is individually rational and not blocked
	\end{itemize}
\end{frame}
\begin{frame}{Deferred Acceptance}
	The deferred acceptance mechanism is a mainstay of this paper. In the man-proposing version of this mechanism, the algorithm iteritavely selects a man $m$ who proposes to his most preferred woman who has not yet rejected him, unless he has been rejected by all women who he prefers to being unmatched.  If she prefers him to her current tenative assignment (and prefers him to being unmatched), she holds his proposal and rejects her current assignment (if one exists). Otherwise, she rejects $m$.  This process repeats until all men are either tentatively matched or have run out of proposals to make, and arrives at a stable matching that is weakly preferred by all men to any other stable matching.  We call this the man-optimal stable match, because it is weakly preferred by all men to any other stable match.  A parallel result exists for the woman-proposing algorithm.

\end{frame}

\begin{frame}{Market Participants}
	
\begin{itemize}
	\item $n$ doctors  indexed $\{ d_1, d_2, ... ,d_n\}$ and  $n$ hospitals  indexed $\{ h_1, h_2, ... ,h_n\}$
	\item The degree to which doctors' preference are random is indexed $k_d$, and for hospitals $k_h$
	\item Preference formation
	\begin{itemize}
		\item Each doctor forms their preference list by first ranking all hospitals according to their index number, so $h_1$ is the doctor's favorite and $h_n$ their least favorite (the doctor's least favorite option is to be unmatched).  
		\item Each $D_d$ randomizes their preference list over $H_{d-k_c},...,H_{d+k_d}$
		\item Hospitals proceed likewise
	\end{itemize}
\end{itemize}
\end{frame}


\begin{frame}{Analyzing the model}
	
\begin{theorem}
    Rank lists in which each doctor $d_d$ ranks $h_{d-\text{Max}(k_d,k_h)-2}, ..., h_{d+\text{Max}(k_d,k_h)+2}$ have the same set of stable matchings as full rank lists. 
\end{theorem}
\begin{proof}
    I had this proof neatly typed, and then my computer ate it.  Nevertheless, I assure you it is a neat and tidy proof in which I have great confidence.
\end{proof}
\end{frame}
\begin{frame}{Analyzing the model}
	
	
	
\begin{conj}
    For an arbitrarily large $n$ the fraction of participants with multiple stable matches is tightly bounded away from zero, and depends primarily on the difference between $k_d$ and $k_h$.
\end{conj}
\begin{figure}
    \includegraphics[width=.7 \textwidth]{block_size.png}
    \caption{Core Size}
  \end{figure}

\end{frame}

\begin{frame}{Analyzing the model}
    The charachtarization of the above simulation result should be a combination of the proof from \cite{Ashlagi2017} and from \cite{Immorlica2005}
\end{frame}

\begin{frame}{Bibliography}
\bibliography{../../library}
\bibliographystyle{aea}
\end{frame}



\end{document}
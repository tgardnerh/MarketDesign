% AER-Article.tex for AEA last revised 22 June 2011
\documentclass[WP]{AEA}

% The mathtime package uses a Times font instead of Computer Modern.
% Uncomment the line below if you wish to use the mathtime package:
%\usepackage[cmbold]{mathtime}
% Note that miktex, by default, configures the mathtime package to use commercial fonts
% which you may not have. If you would like to use mathtime but you are seeing error
% messages about missing fonts (mtex.pfb, mtsy.pfb, or rmtmi.pfb) then please see
% the technical support document at http://www.aeaweb.org/templates/technical_support.pdf
% for instructions on fixing this problem.

% Note: you may use either harvard or natbib (but not both) to provide a wider
% variety of citation commands than latex supports natively. See below.

% Uncomment the next line to use the natbib package with bibtex 
%\usepackage{natbib}
\usepackage{hyperref}
% Uncomment the next line to use the harvard package with bibtex
\usepackage[abbr]{harvard}

\usepackage{ amssymb }


% This command determines the leading (vertical space between lines) in draft mode
% with 1.5 corresponding to "double" spacing.
\draftSpacing{1.5}

\newtheorem{thm}{Theorm}
\newtheorem{prop}{Proposition}
\newtheorem{lemma}{Lemma}
\newtheorem{deff}{Definition}
\newtheorem{conj}{Conjecture}

\begin{document}

\title{A Theory Paper on Rank List Length}
\shortTitle{List Length}
\author{Tyler Hoppenfeld}
\date{\today}
\pubMonth{Month}
\pubYear{Year}
\pubVolume{Vol}
\pubIssue{Issue}
\JEL{}
\Keywords{}

\begin{abstract}
	In its current iteration, this paper advances three avenues of thought on the size of the set of equilibria in matching markets, seeking to expand on the work by Immorlica and others who argue persuasively that rank list length is intimately related to core size. In section
	 \ref{sect:counterexample}
	 I develop an example where agents are content with short rank lists, preferences are highly correlated, and yet a large fraction of agents have more than one equilibrium allocation.  This section is well developed.  In section
	  \ref{sect:partition},
	  I develop a framework to partition markets in a way that I hope will shed light on why some markets have small equilibria sets, and others large equilibria sets. This section is a work in progress--there are a number op unproven conjectures remaining, and the algorithm may still need modification to generate the desired properties. Finally, I advance the conjecture that in a setting where players may defect from the allocation provided by the mechanism, imposing short rank lists cannot shrink the equilibrium set without giving some players an incentive to defect from the market allocation.
\end{abstract}


\maketitle


\section{Introduction}

Matching markets without transfers are a very small part of the measured economy (comprising almost exclusively kidneys and marriage), however they are isomorphic to matching markets with fixed transfers (eg, the market for resident physicians), and close to isomorphic to the broader labor market, where cash transfers are nearly secondary to the identity of the firm, employee, and job description {Citation Needed}.  Centralized matching markets themselves encompass only a small fraction of the labor market, however the study of centralized matching markets plausibly sheds light on the broader matching of employees to jobs. Furthermore, there are a variety of markets, including nearly every academic job market and most entry-level job markets for new college graduates, where there is no obvious reason why a centralized once-a-year clearing house could not work, and yet do not have this structure. A better understanding of the strengths and limits of centralized markets could help us understand what inefficiencies exist in these markets, whether more efficient solutions are possible, and possibly allow for the design of more efficient markets. 

Theory tells us that there is no matching mechanism that reliably produces a stable outcome wherein both sides have the incentive to report their preferences accurately (Roth 1985), however in practice many matching markets persist with both sides (apparently) reporting their preferences honestly  (Roth and Peranson 1999).  This result can be explained by the fact that in many of these markets the set of stable matchings is actually quite small.  For example, in the National Resident Matching Program (NRMP) matching of recently graduated physicians to resident positions, all but approximately 10 of 20,000 participants have a unique partner in the set of all stable matchings (Roth and Peranson 1999), and so in all but a very small number of cases any misrepresentation of preferences would lead to a less-prefered outcome. Roth and Peranson posit that the small set of stable matchings (SSM) of these markets is likely driven by the correlated nature of preferences, and by the fact that for highly correlated preferences, the SSM is very small.  In a parallel argument,  they note that finite preference lists, as is the case with the NRMP, lead to a very small SSM as well.  

The concept of stable matchings is intuitively appealing in the same way the concept of the core is, but it also has empirical support as an equilibrium. In a review of systems that assign trainee physicians to hospitals in Great Brittan, the markets that produced stable matchings persisted with broad participation, while the markets that did not produce stable matchings failed, with large numbers of applicants and hospitals preempting the market and contracting before the market (Roth 2003).
(Roth and Peranson 1999) use simulation to demonstrate these results, and (Immorlica and Mahdian 2005) and then (Kojima and Pathak 2009) build on this with the theoretical result that as the number of market participants grows, the SSM becomes arbitrarily small as a fraction of market size.  

This paper extends these theoretical results to examine short rank lists in a setting that allows for more realistic preference formation and short rank lists. In this setting I show that short rank lists are not sufficient to ensure a small SSM.  Further, I characterize a partition over the universe of agents, and explore how that partition can be used to understand the size of the SSM and the magnitude of the difference in utility between the best and worst stable matchings an agent can receive.


\section{Model}
\subsection{Preliminary Definitions}

Let there be $M$ men and $W$ women, who together form the univgerse of participants $U = W \cup M$.  Each man has a strict preference relation over the set of women $\succ_{m}$, and women have likewise have preferences $\succ_w$ over men. A market is tuple $\Gamma \ (M,W, \succ_{M}, \succ_W)$ where $\succ_{M} = (\succ_{m})_{m \in M}$ and $\succ_W = (\succ_w)_{w \in W}$.

A matching $\phi$ is a mapping from $U$ onto itself such that (i) for every $m$, $|\phi(m)| = 1$, $\phi(m) \in W$ and for every  $w$, $|\phi(w)| = 1$, $\phi(w) \in M$, (ii) $\phi(m) = w$ iff $\phi(w) = m$.  That is, a matching is simply a one-to-one assignment of men to women.

We say a matching $\phi$ is blocked by a pair $\{m,w\}$ if $w \succ_m \phi(m)$ and $m \succ_w \phi(w)$.  A matching is stable if it is not blocked.

The deferred acceptance mechanism is a mainstay of this paper. In the man-proposing version of this mechanism, the algorithm iteritavely selects a man $m$ who proposes to his most preferred woman who has not yet rejected him.  If she prefers him to her current tenative assignment, she holds his proposal and rejects her current assignment (if one exists). Otherwise, she rejects $m$.  This process repeats until all men are matched, and arrives at a stable matching that is weakly preferred by all men to any other stable matching.  We call this the man-optimal stable match.  A paralell result exists for the woman-proposing algorithm.

\section{Analysis of small perturbations from perfect correlation}  \label{sect:counterexample}
This section develops a proof technique to build a plausible example of a large market where preferences are highly correlated, rank lists can be short, and the market has multiple stable equilibria in the large market limit. 

\subsection{Example }
The following is an example in which the two sides of the market are nearly perfectly correlated, rank lists can be short, but a constant non-zero fraction of the market has more than one stable matching.
\begin{itemize}
	\item To construct this market, first consider countable men indexed $\{ m_0, m_1, ...\}$ and women indexed $\{ w_0, w_1, ...\}$
	
	\item Now, to begin, assign preferences such that, for each man $m_i$, woman $w_j$ is preferred to woman $w_{j'}$ iff $j' > j$, and for each woman $w_j$, man $m_i$ is preferred to woman $m_{i'}$ iff $i' > i$.  In other words, all women agree that man $m_0$ is most preferred, and so on. 
	

	
	\item Next, we perturb the preferences of men by the following process.  All entries on each man's preference list are simultaneously exposed to a Poisson process, with an expected success rate of $\epsilon$, for one unit of time. If the entry for woman $w_i$ on the preference list of man $m_j$ experiences a Poisson success, he switches her with the woman she is immediately preferred to on his preference list, unless she is already his first choice. (I.e. he promotes her by one position on his rank list).
	
	
	For a given matching $\phi$ we categorize perturbations to a man $m_j$'s rank list as follows:
	\begin{itemize}
		\item Upward: The Poisson process hits the entry for $\phi(m_j)$

		
		\item Downward: The Poisson process hits the entry in $m_j$'s list immediately dis-preferred to $\phi(w_j)$
		\item Irrelevant: The Poisson process hits any other entry 
		
	\end{itemize}
	
	Now we repeat the perturbation process for the women, applying the same categorizations.
	

\end{itemize}


\subsection{Framework of the math}

Now I develop a framework to analyze the above example
\begin{deff}
	Next Preferred Available Woman (NPAW): each man's NPAW is the man's most preferred woman who is neither currently his match, nor is matched to a man she prefers to him. 
\end{deff}

\begin{deff}
	Immediately preferred: $m_i$ is immediately preferred to $m_{i'}$ by woman $w_j$ if she prefers  $m_{i}$ to $m_{i'}$ and there is no other man $m_{i''}$ such that she prefers $m_{i''}$ to $m_{i'}$, but not $m_{i}$.
\end{deff}



\begin{lemma}
	Consider the stable matching $\phi$ under complete preferences. This algorithm will terminate at  $\phi^w$, the woman optimal stable matching.
	\begin{enumerate}
		\item Each man points to the current partner of his NPAW
		\item Identify any cycles
		\item \label{new_match} In each cycle, each man re-matches with his NPAW
		\item repeat until no more cycles are identified
	\end{enumerate}
\end{lemma}
\begin{proof}
	\begin{enumerate}
		\item 	First we show by contradiction that the matching after step \ref{new_match} is stable (call this matching $\phi$):
		\begin{enumerate}
			\item \label{proof_w_pref_m_to_phi} $ m \succ_w \phi(w)$ 
			\begin{itemize}
				\item Suppose there exists a blocking pair $\{m,w\}$ to the new match $\phi '$.
				\item  $ m \succ_w \phi'(w)$ as ${m,w}$ blocks $\phi'$
				\item $ \phi'(w) \succsim_w \phi(w)$ by construction
				\item thus $ m \succ_w \phi(w)$ 
			\end{itemize}
			\item $ \phi(m) \succ_m w \succ_m \phi'(m)$
			\begin{itemize}
				\item $\phi(m) \succ_m w$  by combing   conclusion \ref{proof_w_pref_m_to_phi} with the fact that ${m,w}$ does not block $\phi$ 
				\item $w \succ_m \phi'(m)$
			\end{itemize}
		This is a contradiction, since $\phi'(m)$ is $m$'s NPAW in matching $\phi$
		\end{enumerate}	
		\item Now, we show by contradiction that the algorithm terminates on the woman-optimal stable matching
		\begin{itemize}
			\item Suppose the algorithm stops at stable matching $ \phi ' \neq \phi ^w$, and call $M_c$ the set of agents for whom $ \phi '(m) \neq \phi ^w(m)$ or $\phi'(w) \neq \phi^w(w)$
			\item Since the algorithm has terminated, there is some man $m \in M_c$ whose NPAW is not in  $M_c$ (otherwise there would be a cycle).  
			\item now consider $w = \phi^w(m)$: Since $\phi^w$ is woman optimal and man pessimal, $ \phi'(m)\succ_m w$ and $m \succ_w \phi'(w)$
			\item since $w$ is in $M_c$, she is not $m$'s NPAW, so he must have some other NPAW $w^*$ for whom:
			\begin{itemize}
				\item $m \succ_{w^*} \phi'(w^*)$
				\item $\phi'(m) \succ_m w^* \succ_m w$
				\item $\phi'(w^*) = \phi^w(w^*)$
			\end{itemize} 
			\item Thus $\{m,w^*\}$ blocks $\phi^w$, a contradiction
		\end{itemize}
	\end{enumerate}
\end{proof}

\subsection{Putting it together}
\begin{itemize}

	\item First consider man proposing DA with $\epsilon = 0$. Each man $m_i$'s NPAW is $m_{i+1}$, so there are no cycles, and there is exactly one stable matching

	\item Now consider if man $m_i$ has had an upwards perturbation, and $w_{i-1}$ has a downward perturbation. The relevant preference lists are given below, with the man-optimal stable matchings underlined:
	\begin{itemize}
		\item $m_i$: $\{...,w_{i-2}, \underline{w_i}, w_{i-1},...\}$
		\item $w_{i-1}$: $\{...,\underline{m_i}, m_{i-1}, m_{i+1}, ...\}$
		\item $m_{i-1}$: $\{...,\underline{w_{i-1}}, w_i, ...\}$
		\item $w_i$: $\{...,m_{i-1}, \underline{m_i},...\}$
	\end{itemize} 
	We see that $m_i$ and $m_{i-1}$ are each matched to their NPAW, inducing a cycle and thus multiple stable matches.
	\item Next consider if man $m_i$ has had a downward perturbation and woman $w_{i+1}$  experiences an upwards perturbation. The relevant   preference lists are given below, with the man-optimal stable matchings underlined:
	\begin{itemize}
		\item $m_i$: $\{...,w_{i-1}, \underline{w_{i+1}}, w_{i},...\}$
		\item $w_{i}$: $\{...,m_i,\underline{m_{i+1}}, ...\}$
		\item $w_{i+1}$: $\{...,m_{i-1}, m_{i+1}, \underline{m_i},...\}$
		\item $m_{i+1}$: $\{...,\underline{w_{i}}, w_{i+1}, ...\}$

	\end{itemize} 
		We see that $m_i$ and $m_{i+1}$ are each matched to their NPAW, inducing a cycle and thus multiple stable matches.
	
	\item  For small $\epsilon$ values,  each man expects to be involved in one of these cycles with probability $2 \epsilon^2$, regardless of the size of the market.
	
	\item Finally, since perturbations are rare, agents are nearly guaranteed to match within a very small range around the match they would have if $\epsilon = 0$, so  a short rank list is equivalent to listing complete preferences. 
\end{itemize}




\section{An Algorithm to Partition the Market} \label{sect:partition}
This is a related approach: I conjecture that it will be conceptually valuable to partition the market into ``segments'' that are functionally separate .  To do this, 


Call the universe of unmatched participants $U$, the first choice of man $m$ among the unmatched women $p_1(m)$, and likewise for women.


To to find the first partition, initialize each $m_i$ with the attribute $m_i.R = \emptyset$. Then choose arbitrary $m$, then select $w = p_1(m)$, $m' = p_1(w)$, and so on, until a cycle is found. Call the members of that cycle $C^*$. Initialize counter $i = 1$  The algorithm proceeds as follows: 


\begin{enumerate}
	\item for each man $m$ in $C^*$, add $p_1(m)$ to $m.R$ (likewise for women)
	\item  \label{seg_alg:comp} for each man $m$ in $U$, add to $m.R$ as needed so that for any $m$ and $w$, if $m \in w.R$, then $w \in m.R$  (likewise for women)
	\item  \label{seg_alg:ext} for each man $m$ in $U$ such that $m.R \neq \emptyset$, call  $w_p$ his least preferred woman in $m.R$, and add to $m.R$ such that if $w \succ_{m} w_p$, then $w \in m.R$ (likewise for women)
	\item if any additions are made to any $m.R$ or $w.R$ in step \ref*{seg_alg:ext}, return to step \ref*{seg_alg:comp}, otherwise proceed
	\item call $C_0$ the set of all agents for whom $m.R$ or $w.R$ is nonempty
	\item  call $\phi_{m0}$ the man optimal stable matching on $C_0$
\end{enumerate}

To find the next partition, remove all men and women matched in $\phi_{m0}$ from $U$, and repeat the entire process to find $C_1$ and $\phi_{m1}$

\begin{lemma}
	$\phi_{mi}$ is a re-arrangement of the matched agents in $\phi_{wi}$
\end{lemma}
\begin{proof}
	This is an application of the rural hospitals theorem
\end{proof}

\begin{prop}
	$\phi_m = \{\phi_{m0}, \phi_{m1}, ...\}$ is identical to the man proposing deferred acceptance matching $\Phi_m $
\end{prop}
\begin{proof}
	First we prove this for $\phi_{m0}$, proceeding by contradiction.  Call $w = \phi_{m0}(m) \neq \Phi_m(m)$
	
	Either $m'\succ_w m$ or $m \succ_w m'$. We will handle each case separately.
	
	First suppose $m'\succ_w m$
	\begin{itemize}
		\item $m'$ must never have proposed to $w$, and so must be matched to $w' = \phi_{m0}(m)$ where $w' \succ w$
		\item call $m'' = \Phi_m (w')$
		\item as the pair $\{m', w'\}$ do not block $\Phi_m$, $m'' \succ_{w'} m'$
		\item we now have a sequence $\{m,w,m',w',m'',...\}$ contained within the finite set $C$, and thus we have a cycle such that each man prefers the woman he is matched with in  $\phi_{m0}$, and each women prefers her match in $\Phi_m$
		\item this cycle represents a pareto improvement over  $\Phi_m$ for men, and thus there must be an interrupter in $U \setminus C$ such that he forms a blocking pair with one of the women in this cycle. 
		\item the man should have been added to $C$ at step \ref*{seg_alg:ext}, thus a contradiction
	\end{itemize}
	Now suppose  $m \succ_w m'$
 	\begin{itemize}
 		\item call $w^a = \Phi_m(m)$
 		\item as $\{m,w\}$ do not block $\phi_{m0}$, $w^a \succ_{m} w$
 		\item call $m^a = \phi_{m0}(w^a)$
 		\item now as $\{m,w^a\}$ do not block $\Phi_m$, $m^a \succ_{w^a} m $
 		\item we now have a cycle where $\Phi_m$ is a pareto improvement over $\phi_{m0}$ for men
 		\item this implies an interrupter in $C \setminus U$, which is an empty set, a contradiction.
 	\end{itemize}
	Thus $\phi_{m0}$ is a subset of $\Phi_m$
	
We now note that after removing $\phi_{m0}$ from U, the same logic holds, and the proof continues by induction. 
\end{proof}
\begin{prop}
	If there is a man $m \in \{C_i  \setminus \phi_i\}$ then there is no woman $w \in \{C_i  \setminus \phi_i\}$
\end{prop}
\begin{proof}
	This is obvious as all participants rank all other participants
\end{proof}
	
\begin{conj}
	A non-trivial partition of the market has more than one stable allocation if it is balanced
\end{conj}
	each man is inclrded because he was prefered to some woman's match in the mpda matching.  thus he has a NPAW.  If each man has an NPAW, then there is a cycle, and thus an alternats stable matching.
\begin{conj}
		A partition of the market has only one stable allocation if it is not balanced
\end{conj}
	

\section{A Conjecture on short rank lists}
\begin{conj}
	 In a setting where players may defect from the allocation provided by the mechanism, imposing short rank lists cannot shrink the equilibrium set without giving some players an incentive to defect from the market allocation.
\end{conj}
\begin{proof}
	The basic idea here is that short rank lists shrink the Equilibrium set by forcing agents to leave unranked (ie declare unacceptable) matches that they would have in their less prefered allocations.  If they have enough information to choose this cutoff optimally, then they wouldnt need to be forced to use a short rank list (they would do so anyway).  If, however, they don't have very good information, then they will sometimes be unmatched, and be able to induce someone to defect from the allocation with them. 
\end{proof}

\bibliographystyle{aea}
\bibliography{library}

% The appendix command is issued once, prior to all appendices, if any.
\appendix



\section{Mathematical Appendix}


\end{document}


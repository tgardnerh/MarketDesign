% AER-Article.tex for AEA last revised 22 June 2011
\documentclass[WP]{AEA}

% The mathtime package uses a Times font instead of Computer Modern.
% Uncomment the line below if you wish to use the mathtime package:
%\usepackage[cmbold]{mathtime}
% Note that miktex, by default, configures the mathtime package to use commercial fonts
% which you may not have. If you would like to use mathtime but you are seeing error
% messages about missing fonts (mtex.pfb, mtsy.pfb, or rmtmi.pfb) then please see
% the technical support document at http://www.aeaweb.org/templates/technical_support.pdf
% for instructions on fixing this problem.

% Note: you may use either harvard or natbib (but not both) to provide a wider
% variety of citation commands than latex supports natively. See below.

% Uncomment the next line to use the natbib package with bibtex 
%\usepackage{natbib}
\usepackage{hyperref}
% Uncomment the next line to use the harvard package with bibtex
\usepackage[abbr]{harvard}

\usepackage{ amssymb }


% This command determines the leading (vertical space between lines) in draft mode
% with 1.5 corresponding to "double" spacing.
\draftSpacing{1.5}

\newtheorem{thm}{Theorm}
\newtheorem{prop}{Proposition}
\newtheorem{lemma}{Lemma}
\newtheorem{deff}{Definition}
\newtheorem{conj}{Conjecture}

\begin{document}

\title{A Theory Paper on Rank List Length}
\shortTitle{List Length}
\author{Tyler Hoppenfeld}
\date{\today}
\pubMonth{Month}
\pubYear{Year}
\pubVolume{Vol}
\pubIssue{Issue}
\JEL{}
\Keywords{}

\begin{abstract}
Your abstract here.
\end{abstract}


\maketitle

\begin{abstract}
	An abstract goes here
\end{abstract}

\section{Introduction}

%%Matching markets without transfers are a very small part of the measured economy (comprising almost exclusively kidneys and marriage), however they are isomorphic to matching markets with fixed transfers (eg, the market for resident physicians), and close to isomorphic to the broader labor market, where cash transfers are nearly secondary to the identity of the firm, employee, and job description {Citation Needed}.  Centralized matching markets themselves encompass only a small fraction of the labor market, however the study of centralized matching markets plausibly sheds light on the broader matching of employees to jobs. Furthermore, there are a variety of markets, including nearly every academic job market and most entry-level job markets for new college graduates, where there is no obvious reason why a centralized once-a-year clearing house could not work, and yet do not have this structure. A better understanding of the strengths and limits of centralized markets could help us understand what inefficiencies exist in these markets, whether more efficient solutions are possible, and possibly allow for the design of more efficient markets. 
%To begin studying matching markets, we first want to define the structure. We generally study a market with two distinct types of agents, which we colloquially call “Men” and “Women” in the one-to-one setting, or “Doctors” and “Hospitals” in the one-to-many setting. Since paper will model the one-to-one market with  participants of each type, , each of whom has complete and transitive preferences over being matched to each participant of the other type, or being left unmatched (which we will denote as being matched to oneself). Define a matching  such that  iff , , and .  This is to say, a matching is an outcome of the market that is mechanically possible. 
%The equilibrium concept generally used in the literature is that of stability, which, in the absence of transfers, is identical to the core in the one-to-one setting, and a stricter version of equilibrium than the core in the many-to-one setting {citation needed}.  We say a matching  is stable if:
%For each agent matched to another agent of the other type, they prefer to be matched with that agent rather than being unmatched
%There does not exist a “blocking pair ” such that   prefers  to , and  prefers  to 

Theory tells us that there is no matching mechanism that reliably produces a stable outcome wherein both sides have the incentive to report their preferences accurately (Roth 1985), however in practice many matching markets persist with both sides (apparently) reporting their preferences honestly  (Roth and Peranson 1999).  This result can be explained by the fact that in many of these markets the set of stable matchings is actually quite small.  For example, in the National Resident Matching Program (NRMP) matching of recently graduated physicians to resident positions, all but approximately 10 of 20,000 participants have a unique partner in the set of all stable matchings (Roth and Peranson 1999), and so in all but a very small number of cases any misrepresentation of preferences would lead to a less-prefered outcome. Roth and Peranson posit that the small set of stable matchings (SSM) of these markets is likely driven by the correlated nature of preferences, and by the fact that for highly correlated preferences, the SSM is very small.  In a parallel argument,  they note that finite preference lists, as is the case with the NRMP, lead to a very small SSM as well.  

The concept of stable matchings is intuitively appealing in the same way the concept of the core is, but it also has empirical support as an equilibrium. In a review of systems that assign trainee physicians to hospitals in Great Brittan, the markets that produced stable matchings persisted with broad participation, while the markets that did not produce stable matchings failed, with large numbers of applicants and hospitals preempting the market and contracting before the market (Roth 2003).
(Roth and Peranson 1999) use simulation to demonstrate these results, and (Immorlica and Mahdian 2005) and then (Kojima and Pathak 2009) build on this with the theoretical result that as the number of market participants grows, the SSM becomes arbitrarily small as a fraction of market size.  

I intend to expand on this prior work by proposing a more flexible model of the underlying preference structure.

In practice, medical students applying to residency submit a rank list that is reports their preferences among the plausibly achievable subset of residency programs. The residency application process begins with a round of interviews, where programs extend interview invitations to a strategically chosen set of applicants, and applicants may choose to attend interviews at a subset of programs that extended invitations. Since the marginal cost of applying to a program is zero (http://www.nrmp.org/match-fees/) medical students applying to residency apply broadly, and so as a result of the interview process, students and programs rank a subset of mutually plausible matches, but their rank lists may omit an arbitrarily large number of potential suitors by whom they have already been rejected.

Even if it is generally true that short rank li

\subsection{Notes to myself--most of this won't make the final cut}
Immorlica Preferences
I now will describe the rank list construction method used by (Immorlica and Mahdian 2005) (which i will henceforth refer to as Immorlica preferences). 
•	A distribution of hospitals is fixed (Uniform distribution for (Immorlica and Mahdian 2005), arbitrary distribution for (Kojima and Pathak 2009))
Each doctor draws a hospitals from the distribution (iid) until k hospitals have been chosen
these k hospitals form the rank list, in order of their draw

These preferences have no basis in utility theory, and I conjecture some non-intuitive properties which I intend to prove.	
In the case of the large-market limit, the distribution of preferences will be roughly uniform (ie, no generalized preferences between hospitals), or the nearly all rank lists will be dominated by a small number of hospitals.
•	To understand this, first note that, for all but a finite and fixed number of hospitals, the likelihood of any one student ranking that hospital goes to zero as the market grows without bound. (This is an application of the pidgin-hole theorem, I believe).
•	Next note that ranking hospital i but not j implies that i is preferred to j.  
•	This means that, for almost any hospital i and j, the student is almost certain to prefer i to j, conditional on their ranking i, and vise-versa
•	This is conceptually very close to saying that there are no generalized preferences between almost any hospitals.
•	If, however, a finite number of institutions occupy a finite portion of the distribution as the market grows without bound, then those hospitals will become massively oversubscribed, and the likelihood of any one applicant matching there is zero, conditional on that applicant ranking the hospital.
Almost all hospitals are unacceptable to almost every medical student.  This is, of course, practically true for every short rank list, however we aim to understand whether the allocation is stable under the true preferences, not just the reported preferences.  Most medical students do not rank any of the top ten hospitals, however this is presumably not because they find all of the top hospitals unacceptable, but instead simply that they find them unobtainable.  In the case of Immorlica preferences, many applicants who do not rank the “top” hospitals would be able to match there, if only they had ranked those programs (this is a conjecture, but I strongly believe it to be true).


\subsection{Analysis of small perturbations from perfect correlation}
\begin{deff}
	Next Preferred Available Woman (NPAW): each man's NPAW is the man's most preferred woman who is neither currently his match, nor is matched to a man she prefers to him. 
\end{deff}

\begin{deff}
	Immediately preferred: $m_i$ is immediately preferred to $m_{i'}$ by woman $w_j$ if she prefers  $m_{i}$ to $m_{i'}$ and there is no other man $m_{i''}$ such that she prefers $m_{i''}$ to $m_{i'}$, but not $m_{i}$.
\end{deff}



\begin{lemma}
	Consider the stable matching $\phi$. This algorithm will terminate at  $\phi^w$, the woman optimal stable matching.
	\begin{enumerate}
		\item Each man points to the current partner of his NPAW
		\item Each woman points to the woman she currently holds
		\item \label{new_match} In each cycle, each man re-matches with the woman he is pointed to
	\end{enumerate}
\end{lemma}
\begin{proof}
	\begin{enumerate}
		\item 	First we show by contradiction that the matching after step \ref{new_match} is stable:
		\begin{itemize}
			\item Suppose there exists a blocking pair $\{m,w\}$ to the new match $\phi '$. 
			\item By construction:
					\begin{itemize}
						  \item $m \succ_w \phi '^{-1}(w) \succsim_w  \phi ^{-1}(w)$
						  \item  $w \succ_m \phi'(m)$
						  \item $\phi(m) \succsim_m \phi'(m)$
					\end{itemize}
			\item As $\{m,w\}$ do not block $\phi$, we see that $\phi(m) \succsim_m w \succ_m \phi'(m)$
			\item this is a contradiction, since $ \phi'(m)$ was $m$'s NPAW

		\end{itemize}	
		\item Now, we show by contradiction that the algorithm terminates on the woman-optimal stable matching
		\begin{itemize}
			\item Suppose the algorythm stops at stable matching $ \phi ' \neq \phi ^w$, and call $M_c$ the set of men for whom $ \phi '(m) \neq \phi ^w(m)$
			\item Since the algorythm has terminated, there is some man $m \in M_c$ for whom $w = \phi'^{-1} (NPAW_m) \notin  M_c$ 
			\item By construction:
			\begin{itemize}
					\item  $w \succ_m \phi^w(m)$
					\item  $m \succ_w \phi '^{-1}(w) \succsim_w  \phi^{w^{-1}}(w)$
			\end{itemize}
			\item Thus $\{m,w\}$ blocks $\phi^w$ 
		\end{itemize}
	\end{enumerate}
\end{proof}

\subsection{Example }
The following is an example in which the two sides of the market are nearly perfectly correlated, rank lists can be short, but a constant non-zero fraction of the market has more than one stable matching.
\begin{itemize}
	\item To construct this market, first consider countable men indexed $\{ m_0, m_1, ...\}$ and women indexed $\{ w_0, w_1, ...\}$
	
	\item Now, to begin, assign preferences such that, for each man $m_i$, woman $w_j$ is preferred to woman $w_{j'}$ iff $j' > j$, and for each woman $w_j$, man $m_i$ is preferred to woman $m_{i'}$ iff $i' > i$.  In other words, all women agree that man $m_0$ is most preferred, and so on. 
	
	\item Note that, at this stage, if we run man-optimal DA, each man $m_i$'s NPAW is $m_{i+1}$, so there are no cycles, and there is exactly one stable matching
	
	\item Next, we perturb the preferences of men by the following process.  Each woman on each man's preference list is exposed to a Poisson process with an expected success count of $\epsilon$ in which a success causes the woman to be switched with the woman she is immediately preferred to.
	
	\begin{itemize}
		\item If the immediate preference relation immediately above woman $w_{i-1}$  and $w_i$ are reversed in man $m_i$'s preference list, Call this an "upward" perturbation
		
		\item If the immediate preference relation below woman $w_i$ is perturbed, call this a "downward" perturbation
		
	\end{itemize}

	Now we repeat the perturbation process for the women. 
	
	\begin{itemize}
		\item First consider if man $m_i$ has had an upwards perturbation.  His next preferred woman is now $w_{i-1}$, while his next preferred achievable woman is still $w_{i+1}$. If woman $w_{i-1}$ adjusts her preferences such that $m_i$ is immediately preferred to $m_{i-1}$, a cycle is induced. Otherwise, no no other preference change will induce a cycle for man $m_i$
		\item Now consider if man $m_i$ has had a downwards perturbation. $m_i$ is now matched with $w_{i+1}$ and $m_i$'s next preferred woman is now his next preferred achievable woman, is $w_i$. $m_{i+1}$ is now matched to $w_i$,  $m_{i+1}$'s next preferred woman is $w_i$, and his next preferred achievable woman is $w_{i+2}$.   In this case, a cycle is induced if $m_i$ and $m_{i+1}$ are reversed in the preference list of $w_{i+1} $. 
	\end{itemize}

	\item  For small $\epsilon$ values, each man expects $2\epsilon$ perturbations.  As there is exactly one perturbation to the woman's preferences that would complete the cycle, then each man expects to be involved in a cycle  with probability $2 \epsilon^2$, regardless of the size of the market.
	
	\item Finally, since perturbations are rare, a 3 element rank list is equivalent to listing complete preferences.
\end{itemize}




\section{Goals of this paper}
In this paper I intend to:
\begin{enumerate}
	\item \label{goal:Short_is_beside_point} Prove that an arbitrary rank-list construction either:
	\begin{itemize}
		\item allows for agents to effectively list their complete preferences with a short rank list, or
		\item does not produce a stable allocation if agents are required to submit shortened ranklists
		
	\end{itemize}
	\item Show that requiring market participants to report a shorter rank list than their actual preferences cannot improve the incentive compatibility of the market.  This is a corollary of goal \ref*{goal:Short_is_beside_point}
	
	\item \label{goal:alternate_pref} Propose an alternate construction of large market preferences that is tractable, more realistic, and has the following properties:
	\begin{itemize}
		\item In the limit of large markets correlation of preferences approaches zero
		\item In the limit of large markets, a short preference list is perfectly adequate (ie, almost no participant will, ex post, wish they had been able to report a longer list)
		\item In the limit of large markets, a positive fraction of participants have more than one stable match possible, even with short preference lists
	\end{itemize}


\end{enumerate}


\section{Alternate construction of large market preferences}

I begin by addressing goal \ref*{goal:alternate_pref} because it illustrates and motivates the other two. 

My preferences are constructed as follows:
\begin{enumerate}
	\item There are $n$ participants on each side of the market, and we will parameterize the degree to which preferences are random as $k$.
	\item Each man $i \in \{1, 2, 3, ... , n\}$ ranks all women $j\in \{1, 2, 3, ... , n\}$ in order (woman $j=1$ first, $j=2$ second, etc). Each woman does the converse 
	\item In each man's preference list, a random women in the list is chosen, and moved down the rank list by one position if she is not already at the end.  This is repeated  $k$ times
\end{enumerate}




\section{Simulation Results}

The simulation results anticipate the analytical finding that I hope to attain. 

\begin{itemize}
	\item In a simulation with market size of  1000, 2000, and 10,000 participants,  a $k$ of 10 consistently means that about 40 men are matched to different women in the man-optimal vs woman-optimal matching. 
	\item When $k=n$    about 3/4 participants have more than one potential stable match, regardless of market size ($n= \{30, 100, 200, 1000, 2000, 10000\}$).
	\item If $k$ is not $>>n$, then $|i-j| < 3$ for nearly every match, meaning that a short-rank-list constraint does not bind
\end{itemize}
 

From this, I draw the following conclusions:
\begin{enumerate}
	\item If we keep $k=n$, in the large market the correlation of preferences goes to 1 (because the expected number of moves (up or down) on a rank list is $2k$, which becomes an increasingly small fraction of $n$)
	\item However, the number of participants who have more than one potential stable match grows as approximately $\frac{3}{4} n$ if $k=n$ in the large market
	\item This is true even if we restrict rank lists to be short
	\item Thus, we need a different explanation for a small set of stable matches--neither list length nor preference correlation is adequate
\end{enumerate}


\section{Analytical Efforts}

First I look at the movement of woman $j$ in man $i$'s rank list, and by symitry the converse. In the large market approximation, the number of times a woman is moved on the rank list is a Poisson distribution with expectation $2k/n$.  The probability of a woman moving $m$ positions on the list is $e^{-2k}I_m(2k/n)$, where $I$ is the Bessel function.  see \href{https://www.wolframalpha.com/input/?i=e%5E-10*BesselI%286%2C+10%29}{wolfram alpha}
	
Some simple work with a symbolic evalutator shows that in any given rank list, woman $j$'s expected position is $j$, with a varience of $2k/n$.
	
Now we consider women who have two or more potential stable matches.  Following Immorlica, we note that for each such woman, if she truncates her rank list immediately above her partner in the man-optimal stable match, she will initiate a rejection chain that returns to her with a proposal she prefers to her original match.	
We can now consider such a rejection chain, and see that a rejection chain is a random walk where each step has a positive expected lenght, and a finite standard deviation.  Thus, after a finite length of rejection chain, there is effectively zero chance of looping back to the starting position.  However, in this market, two things are true:
\begin{itemize}
	\item The rejection chain has a finite chance of returning to the initiating woman in the first few steps.  This is the opposite of the key result from Immorlica 2005 and Kojima 2009.
	\item after a finite number of steps, the liklihood of the chain returning to the initating woman is arbitrarily small, since it is a biased random walk. This means that a short rank list is adequite to entirely describe all the relevant preferences (ie, a short rank list requiremnt does not bind).
\end{itemize}


\section{An Algorithm to Partition the Market}

Call the universe of unmatched participants $U$, the first choice of man $m$ among the unmatched women $p_1(m)$, and likewise for women. initialize counter $i = 1$  The algorithm proceeds as follows: 


\begin{enumerate}
	\item \label{seg_alg:start} Call the man with the lowest index number $m^*$, and call the set $C = \{p_1(m^*), p_1(p_1(m^*)), ...\}$ (Note that this will form a cycle, since the set of participenhs is finite)
	\item \label{seg_alg:DA} Call $\phi_m$ the matching produced by man proposing defered acceptence, and $\phi_w$ likewise for woman proposing defered acceptence.  Call the set of all matched prticipants $\phi$ 
	\item \label{seg_alg:add_C} For each man $m \in C$, add  to $C$ each woman he prefers to $\phi_w(m)$, and likewise for each woman $w \in C$ and $\phi_m(w)$
	\item If step \ref{seg_alg:add_C} adds members to $C$, return to step \ref{seg_alg:DA}.  Otherwise proceed
	\item Terminate if $\phi =  \emptyset$, otherwise remove $\phi$ from $U$, rename $\{\phi_m, \phi_w, C\}$ as $\phi_i = \{\phi_{mi}, \phi_{wi}, C_i\}$ , increment $i$ and return to \ref{seg_alg:start}
	
\end{enumerate}

\begin{lemma}
	$\phi_{mi}$ is a re-arrangement of the matched agents in $\phi_{wi}$
\end{lemma}
\begin{proof}
	This is an application of the rural hospitals theorm
\end{proof}

\begin{prop}
	$\phi_m = \{\phi_{m1}, \phi_{m2}, ...\}$ is identical to the man proposing deferred acceptance matching $\Phi_m $
\end{prop}
\begin{proof}
	First we prove this for $\phi_{m1}$
	\begin{itemize}
		\item S'pose not. There is some man $m$ such that $\Phi_m(m) \neq \phi_{m1}(m)$
		\item $w = \Phi_m(m) \in C_1$ because of step \ref{seg_alg:add_C}
		\item Call $m' =  \phi_{m1}(\phi_{m1}(m)) $ and note that $ \phi_{m1}(m') \neq \Phi_m(m')$ 
		\item There is thus a finite cycle of men $\{m, m', m'', ... , m^n\}$ such that $ \phi_{m1}(m') $ rejected $m$ in favor of $m'$, $ \phi_{m1}(m'') $ rejected $m'$ in favor of $m''$, and so on until $ \phi_{m1}(m) $ rejected $m^n$ in favor of $m$.  This is impossible
	\end{itemize}
We now note that after removing $\phi_{m1}$ from U, the same logic holds, and the proof continues by induction. 
\end{proof}
\begin{prop}
	If there is a man $m \in \{C_i  \setminus \phi_i\}$ then there is no woman $w \in \{C_i  \setminus \phi_i\}$
\end{prop}
\begin{proof}
	This is obvious as all participants rank all other participants
\end{proof}
	
\begin{conj}
	A non-trivial partition of the market has more than one stable allocation if it is balanced
\end{conj}
	
\begin{conj}
		A partition of the market has only oneone stable allocation if it is not balanced
\end{conj}
	

\section{A Conjecture that makes this all worthwhile}



\bibliographystyle{aea}
\bibliography{library}

% The appendix command is issued once, prior to all appendices, if any.
\appendix



\section{Mathematical Appendix}


\end{document}


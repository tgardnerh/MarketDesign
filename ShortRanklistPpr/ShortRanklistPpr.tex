% AER-Article.tex for AEA last revised 22 June 2011
\documentclass[WP]{AEA}

% The mathtime package uses a Times font instead of Computer Modern.
% Uncomment the line below if you wish to use the mathtime package:
%\usepackage[cmbold]{mathtime}
% Note that miktex, by default, configures the mathtime package to use commercial fonts
% which you may not have. If you would like to use mathtime but you are seeing error
% messages about missing fonts (mtex.pfb, mtsy.pfb, or rmtmi.pfb) then please see
% the technical support document at http://www.aeaweb.org/templates/technical_support.pdf
% for instructions on fixing this problem.

% Note: you may use either harvard or natbib (but not both) to provide a wider
% variety of citation commands than latex supports natively. See below.

% Uncomment the next line to use the natbib package with bibtex 
%\usepackage{natbib}
\usepackage{hyperref}
% Uncomment the next line to use the harvard package with bibtex
\usepackage[abbr]{harvard}

\usepackage{ amssymb }


% This command determines the leading (vertical space between lines) in draft mode
% with 1.5 corresponding to "double" spacing.
\draftSpacing{1.5}

\newtheorem{thm}{Theorm}
\newtheorem{prop}{Proposition}
\newtheorem{lemma}{Lemma}
\newtheorem{deff}{Definition}
\newtheorem{conj}{Conjecture}

\begin{document}

\title{A Theory Paper on Rank List Length}
\shortTitle{List Length}
\author{Tyler Hoppenfeld}
\date{\today}
\pubMonth{Month}
\pubYear{Year}
\pubVolume{Vol}
\pubIssue{Issue}
\JEL{}
\Keywords{}

\begin{abstract}
	In its current iteration, this paper advances three avenues of thought on the size of the set of equilibria in matching markets, seeking to expand on the work by Immorlica and others who argue persuasively that rank list length is intimately related to core size. In section
	 \ref{sect:counterexample}
	 I develop an example where agents are content with short rank lists, preferences are highly correlated, and yet a large fraction of agents have more than one equilibrium allocation.  This section is well developed.  In section
	  \ref{sect:partition},
	  I develop a framework to partition markets in a way that I hope will shed light on why some markets have small equilibria sets, and others large equilibria sets. This section is a work in progress--there are a number op unproven conjectures remaining, and the algorithm may still need modification to generate the desired properties. Finally, I advance the conjecture that in a setting where players may defect from the allocation provided by the mechanism, imposing short rank lists cannot shrink the equilibrium set without giving some players an incentive to defect from the market allocation.
\end{abstract}


\maketitle


\section{Introduction}

%%Matching markets without transfers are a very small part of the measured economy (comprising almost exclusively kidneys and marriage), however they are isomorphic to matching markets with fixed transfers (eg, the market for resident physicians), and close to isomorphic to the broader labor market, where cash transfers are nearly secondary to the identity of the firm, employee, and job description {Citation Needed}.  Centralized matching markets themselves encompass only a small fraction of the labor market, however the study of centralized matching markets plausibly sheds light on the broader matching of employees to jobs. Furthermore, there are a variety of markets, including nearly every academic job market and most entry-level job markets for new college graduates, where there is no obvious reason why a centralized once-a-year clearing house could not work, and yet do not have this structure. A better understanding of the strengths and limits of centralized markets could help us understand what inefficiencies exist in these markets, whether more efficient solutions are possible, and possibly allow for the design of more efficient markets. 
%To begin studying matching markets, we first want to define the structure. We generally study a market with two distinct types of agents, which we colloquially call “Men” and “Women” in the one-to-one setting, or “Doctors” and “Hospitals” in the one-to-many setting. Since paper will model the one-to-one market with  participants of each type, , each of whom has complete and transitive preferences over being matched to each participant of the other type, or being left unmatched (which we will denote as being matched to oneself). Define a matching  such that  iff , , and .  This is to say, a matching is an outcome of the market that is mechanically possible. 
%The equilibrium concept generally used in the literature is that of stability, which, in the absence of transfers, is identical to the core in the one-to-one setting, and a stricter version of equilibrium than the core in the many-to-one setting {citation needed}.  We say a matching  is stable if:
%For each agent matched to another agent of the other type, they prefer to be matched with that agent rather than being unmatched
%There does not exist a “blocking pair ” such that   prefers  to , and  prefers  to 

%Theory tells us that there is no matching mechanism that reliably produces a stable outcome wherein both sides have the incentive to report their preferences accurately (Roth 1985), however in practice many matching markets persist with both sides (apparently) reporting their preferences honestly  (Roth and Peranson 1999).  This result can be explained by the fact that in many of these markets the set of stable matchings is actually quite small.  For example, in the National Resident Matching Program (NRMP) matching of recently graduated physicians to resident positions, all but approximately 10 of 20,000 participants have a unique partner in the set of all stable matchings (Roth and Peranson 1999), and so in all but a very small number of cases any misrepresentation of preferences would lead to a less-prefered outcome. Roth and Peranson posit that the small set of stable matchings (SSM) of these markets is likely driven by the correlated nature of preferences, and by the fact that for highly correlated preferences, the SSM is very small.  In a parallel argument,  they note that finite preference lists, as is the case with the NRMP, lead to a very small SSM as well.  

%The concept of stable matchings is intuitively appealing in the same way the concept of the core is, but it also has empirical support as an equilibrium. In a review of systems that assign trainee physicians to hospitals in Great Brittan, the markets that produced stable matchings persisted with broad participation, while the markets that did not produce stable matchings failed, with large numbers of applicants and hospitals preempting the market and contracting before the market (Roth 2003).
%(Roth and Peranson 1999) use simulation to demonstrate these results, and (Immorlica and Mahdian 2005) and then (Kojima and Pathak 2009) build on this with the theoretical result that as the number of market participants grows, the SSM becomes arbitrarily small as a fraction of market size.  

%I intend to expand on this prior work by proposing a more flexible model of the underlying preference structure.

%In practice, medical students applying to residency submit a rank list that is reports their preferences among the plausibly achievable subset of residency programs. The residency application process begins with a round of interviews, where programs extend interview invitations to a strategically chosen set of applicants, and applicants may choose to attend interviews at a subset of programs that extended invitations. Since the marginal cost of applying to a program is zero (http://www.nrmp.org/match-fees/) medical students applying to residency apply broadly, and so as a result of the interview process, students and programs rank a subset of mutually plausible matches, but their rank lists may omit an arbitrarily large number of potential suitors by whom they have already been rejected.

%Even if it is generally true that short rank li

For reference I summarize the preference framework Immorlica and others use to show that short rank lists cause a small equilibrium set.
Immorlica Preferences
I now will describe the rank list construction method used by (Immorlica and Mahdian 2005) (which i will henceforth refer to as Immorlica preferences).

\begin{enumerate}
	\item A distribution of hospitals is fixed (Uniform distribution for (Immorlica and Mahdian 2005), arbitrary distribution for (Kojima and Pathak 2009))
	\item Each doctor draws a hospitals from the distribution (iid) until $k$ distinct hospitals have been chosen these $k$ hospitals form the rank list, in order of their draw
	\item Hospitals have fully random preferences over doctors
\end{enumerate}





\section{Analysis of small perturbations from perfect correlation}  \label{sect:counterexample}
This section develops a proof technique to build a plausible example of a large market where preferences are highly correlated, rank lists can be short, and the market has multiple stable equilibria in the large market limit. 

\subsection{Example }
The following is an example in which the two sides of the market are nearly perfectly correlated, rank lists can be short, but a constant non-zero fraction of the market has more than one stable matching.
\begin{itemize}
	\item To construct this market, first consider countable men indexed $\{ m_0, m_1, ...\}$ and women indexed $\{ w_0, w_1, ...\}$
	
	\item Now, to begin, assign preferences such that, for each man $m_i$, woman $w_j$ is preferred to woman $w_{j'}$ iff $j' > j$, and for each woman $w_j$, man $m_i$ is preferred to woman $m_{i'}$ iff $i' > i$.  In other words, all women agree that man $m_0$ is most preferred, and so on. 
	

	
	\item Next, we perturb the preferences of men by the following process.  All entries on each man's preference list are simultaneously exposed to a Poisson process, with an expected success rate of $\epsilon$, for one unit of time. If the entry for woman $w_i$ on the preference list of man $m_j$ experiences a Poisson success, he switches her with the woman she is immediately preferred to on his preference list, unless she is already his first choice. (I.e. he promotes her by one position on his rank list).
	
	
	For a given matching $\phi$ we categorize perturbations to a man $m_j$'s rank list as follows:
	\begin{itemize}
		\item Upward: The Poisson process hits the entry for $\phi(m_j)$

		
		\item Downward: The Poisson process hits the entry in $m_j$'s list immediately dis-preferred to $\phi(w_j)$
		\item Irrelevant: The Poisson process hits any other entry 
		
	\end{itemize}
	
	Now we repeat the perturbation process for the women, applying the same categorizations.
	

\end{itemize}


\subsection{Framework of the math}

Now I develop a framework to analyze the above example
\begin{deff}
	Next Preferred Available Woman (NPAW): each man's NPAW is the man's most preferred woman who is neither currently his match, nor is matched to a man she prefers to him. 
\end{deff}

\begin{deff}
	Immediately preferred: $m_i$ is immediately preferred to $m_{i'}$ by woman $w_j$ if she prefers  $m_{i}$ to $m_{i'}$ and there is no other man $m_{i''}$ such that she prefers $m_{i''}$ to $m_{i'}$, but not $m_{i}$.
\end{deff}



\begin{lemma}
	Consider the stable matching $\phi$ under complete preferences. This algorithm will terminate at  $\phi^w$, the woman optimal stable matching.
	\begin{enumerate}
		\item Each man points to the current partner of his NPAW
		\item Identify any cycles
		\item \label{new_match} In each cycle, each man re-matches with his NPAW
		\item repeat until no more cycles are identified
	\end{enumerate}
\end{lemma}
\begin{proof}
	\begin{enumerate}
		\item 	First we show by contradiction that the matching after step \ref{new_match} is stable (call this matching $\phi$):
		\begin{enumerate}
			\item \label{proof_w_pref_m_to_phi} $ m \succ_w \phi(w)$ 
			\begin{itemize}
				\item Suppose there exists a blocking pair $\{m,w\}$ to the new match $\phi '$.
				\item  $ m \succ_w \phi'(w)$ as ${m,w}$ blocks $\phi'$
				\item $ \phi'(w) \succsim_w \phi(w)$ by construction
				\item thus $ m \succ_w \phi(w)$ 
			\end{itemize}
			\item $ \phi(m) \succ_m w \succ_m \phi'(m)$
			\begin{itemize}
				\item $\phi(m) \succ_m w$  by combing   conclusion \ref{proof_w_pref_m_to_phi} with the fact that ${m,w}$ does not block $\phi$ 
				\item $w \succ_m \phi'(m)$
			\end{itemize}
		This is a contradiction, since $\phi'(m)$ is $m$'s NPAW in matching $\phi$
		\end{enumerate}	
		\item Now, we show by contradiction that the algorithm terminates on the woman-optimal stable matching
		\begin{itemize}
			\item Suppose the algorithm stops at stable matching $ \phi ' \neq \phi ^w$, and call $M_c$ the set of agents for whom $ \phi '(m) \neq \phi ^w(m)$ or $\phi'(w) \neq \phi^w(w)$
			\item Since the algorithm has terminated, there is some man $m \in M_c$ whose NPAW is not in  $M_c$ (otherwise there would be a cycle).  
			\item now consider $w = \phi^w(m)$: Since $\phi^w$ is woman optimal and man pessimal, $ \phi'(m)\succ_m w$ and $m \succ_w \phi'(w)$
			\item since $w$ is in $M_c$, she is not $m$'s NPAW, so he must have some other NPAW $w^*$ for whom:
			\begin{itemize}
				\item $m \succ_{w^*} \phi'(w^*)$
				\item $\phi'(m) \succ_m w^* \succ_m w$
				\item $\phi'(w^*) = \phi^w(w^*)$
			\end{itemize} 
			\item Thus $\{m,w^*\}$ blocks $\phi^w$, a contradiction
		\end{itemize}
	\end{enumerate}
\end{proof}

\subsection{Putting it together}
\begin{itemize}

	\item First consider man proposing DA with $\epsilon = 0$. Each man $m_i$'s NPAW is $m_{i+1}$, so there are no cycles, and there is exactly one stable matching

	\item Now consider if man $m_i$ has had an upwards perturbation, and $w_{i-1}$ has a downward perturbation. The relevant preference lists are given below, with the man-optimal stable matchings underlined:
	\begin{itemize}
		\item $m_i$: $\{...,w_{i-2}, \underline{w_i}, w_{i-1},...\}$
		\item $w_{i-1}$: $\{...,\underline{m_i}, m_{i-1}, m_{i+1}, ...\}$
		\item $m_{i-1}$: $\{...,\underline{w_{i-1}}, w_i, ...\}$
		\item $w_i$: $\{...,m_{i-1}, \underline{m_i},...\}$
	\end{itemize} 
	We see that $m_i$ and $m_{i-1}$ are each matched to their NPAW, inducing a cycle and thus multiple stable matches.
	\item Next consider if man $m_i$ has had a downward perturbation and woman $w_{i+1}$  experiences an upwards perturbation. The relevant   preference lists are given below, with the man-optimal stable matchings underlined:
	\begin{itemize}
		\item $m_i$: $\{...,w_{i-1}, \underline{w_{i+1}}, w_{i},...\}$
		\item $w_{i}$: $\{...,m_i,\underline{m_{i+1}}, ...\}$
		\item $w_{i+1}$: $\{...,m_{i-1}, m_{i+1}, \underline{m_i},...\}$
		\item $m_{i+1}$: $\{...,\underline{w_{i}}, w_{i+1}, ...\}$

	\end{itemize} 
		We see that $m_i$ and $m_{i+1}$ are each matched to their NPAW, inducing a cycle and thus multiple stable matches.
	
	\item  For small $\epsilon$ values,  each man expects to be involved in one of these cycles with probability $2 \epsilon^2$, regardless of the size of the market.
	
	\item Finally, since perturbations are rare, agents are nearly guaranteed to match within a very small range around the match they would have if $\epsilon = 0$, so  a short rank list is equivalent to listing complete preferences. 
\end{itemize}




\section{An Algorithm to Partition the Market} \label{sect:partition}
This is a related approach: I conjecture that it will be conceptually valuable to partition the market into ``segments'' that are functionally separate .  To do this, 


Call the universe of unmatched participants $U$, the first choice of man $m$ among the unmatched women $p_1(m)$, and likewise for women.

To start, initialize each $m_i$ with the attribute $m_i.R = \emptyset$


 initialize counter $i = 1$  The algorithm proceeds as follows: 


\begin{enumerate}
	\item \label{seg_alg:start} Call the man with the lowest index number $m^*$, and call the set $C = \{p_1(m^*), p_1(p_1(m^*)), ...\}$ (Note that this will form a cycle, since the set of participants is finite)
	\item \label{seg_alg:DA} Call $\phi_m$ the matching produced by man proposing deferred acceptance within set $C$ and $\phi_w$ likewise for woman proposing deferred acceptance.  Call the set of all matched participants $\phi$ 
	\item \label{seg_alg:add_C} For each man $m \in C$, add  to $C$ each woman from the universe of agents that he prefers to $\phi_w(m)$, and likewise for each woman $w \in C$ and $\phi_m(w)$
	\item If step \ref{seg_alg:add_C} adds members to $C$, return to step \ref{seg_alg:DA}.  Otherwise proceed
	\item Terminate if $\phi =  \emptyset$, otherwise remove $\phi$ from $U$, rename $\{\phi_m, \phi_w, C\}$ as $\phi_i = \{\phi_{mi}, \phi_{wi}, C_i\}$ , increment $i$ and return to \ref{seg_alg:start}
	
\end{enumerate}

\begin{lemma}
	$\phi_{mi}$ is a re-arrangement of the matched agents in $\phi_{wi}$
\end{lemma}
\begin{proof}
	This is an application of the rural hospitals theorem
\end{proof}

\begin{prop}
	$\phi_m = \{\phi_{m1}, \phi_{m2}, ...\}$ is identical to the man proposing deferred acceptance matching $\Phi_m $
\end{prop}
\begin{proof}
	First we prove this for $\phi_{m1}$
	\begin{itemize}
		\item S'pose not. There is some man $m$ such that $\Phi_m(m) \neq \phi_{m1}(m)$
		\item $w = \Phi_m(m) \in C_1$ because of step \ref{seg_alg:add_C}
		\item Call $m' =  \phi_{m1}(\phi_{m1}(m)) $ and note that $ \phi_{m1}(m') \neq \Phi_m(m')$ 
		\item There is thus a finite cycle of men $\{m, m', m'', ... , m^n\}$ such that $ \phi_{m1}(m') $ rejected $m$ in favor of $m'$, $ \phi_{m1}(m'') $ rejected $m'$ in favor of $m''$, and so on until $ \phi_{m1}(m) $ rejected $m^n$ in favor of       $m$.  This is impossible
	\end{itemize}
We now note that after removing $\phi_{m1}$ from U, the same logic holds, and the proof continues by induction. 
\end{proof}
\begin{prop}
	If there is a man $m \in \{C_i  \setminus \phi_i\}$ then there is no woman $w \in \{C_i  \setminus \phi_i\}$
\end{prop}
\begin{proof}
	This is obvious as all participants rank all other participants
\end{proof}
	
\begin{conj}
	A non-trivial partition of the market has more than one stable allocation if it is balanced
\end{conj}
	each man is inclrded because he was prefered to some woman's match in the mpda matching.  thus he has a NPAW.  If each man has an NPAW, then there is a cycle, and thus an alternats stable matching.
\begin{conj}
		A partition of the market has only one stable allocation if it is not balanced
\end{conj}
	

\section{A Conjecture on short rank lists}
\begin{conj}
	 In a setting where players may defect from the allocation provided by the mechanism, imposing short rank lists cannot shrink the equilibrium set without giving some players an incentive to defect from the market allocation.
\end{conj}
\begin{proof}
	The basic idea here is that short rank lists shrink the Equilibrium set by forcing agents to leave unranked (ie declare unacceptable) matches that they would have in their less prefered allocations.  If they have enough information to choose this cutoff optimally, then they wouldnt need to be forced to use a short rank list (they would do so anyway).  If, however, they don't have very good information, then they will sometimes be unmatched, and be able to induce someone to defect from the allocation with them. 
\end{proof}

\bibliographystyle{aea}
\bibliography{library}

% The appendix command is issued once, prior to all appendices, if any.
\appendix



\section{Mathematical Appendix}


\end{document}

